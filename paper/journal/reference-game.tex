\documentclass[english,,man,floatsintext]{apa6}
\usepackage{lmodern}
\usepackage{amssymb,amsmath}
\usepackage{ifxetex,ifluatex}
\usepackage{fixltx2e} % provides \textsubscript
\ifnum 0\ifxetex 1\fi\ifluatex 1\fi=0 % if pdftex
  \usepackage[T1]{fontenc}
  \usepackage[utf8]{inputenc}
\else % if luatex or xelatex
  \ifxetex
    \usepackage{mathspec}
  \else
    \usepackage{fontspec}
  \fi
  \defaultfontfeatures{Ligatures=TeX,Scale=MatchLowercase}
\fi
% use upquote if available, for straight quotes in verbatim environments
\IfFileExists{upquote.sty}{\usepackage{upquote}}{}
% use microtype if available
\IfFileExists{microtype.sty}{%
\usepackage{microtype}
\UseMicrotypeSet[protrusion]{basicmath} % disable protrusion for tt fonts
}{}
\usepackage{hyperref}
\hypersetup{unicode=true,
            pdftitle={A communicative framework for early word learning},
            pdfauthor={Benjamin C. Morris~\& Daniel Yurovsky},
            pdfkeywords={keywords},
            pdfborder={0 0 0},
            breaklinks=true}
\urlstyle{same}  % don't use monospace font for urls
\ifnum 0\ifxetex 1\fi\ifluatex 1\fi=0 % if pdftex
  \usepackage[shorthands=off,main=english]{babel}
\else
  \usepackage{polyglossia}
  \setmainlanguage[]{english}
\fi
\usepackage{graphicx,grffile}
\makeatletter
\def\maxwidth{\ifdim\Gin@nat@width>\linewidth\linewidth\else\Gin@nat@width\fi}
\def\maxheight{\ifdim\Gin@nat@height>\textheight\textheight\else\Gin@nat@height\fi}
\makeatother
% Scale images if necessary, so that they will not overflow the page
% margins by default, and it is still possible to overwrite the defaults
% using explicit options in \includegraphics[width, height, ...]{}
\setkeys{Gin}{width=\maxwidth,height=\maxheight,keepaspectratio}
\IfFileExists{parskip.sty}{%
\usepackage{parskip}
}{% else
\setlength{\parindent}{0pt}
\setlength{\parskip}{6pt plus 2pt minus 1pt}
}
\setlength{\emergencystretch}{3em}  % prevent overfull lines
\providecommand{\tightlist}{%
  \setlength{\itemsep}{0pt}\setlength{\parskip}{0pt}}
\setcounter{secnumdepth}{0}
% Redefines (sub)paragraphs to behave more like sections
\ifx\paragraph\undefined\else
\let\oldparagraph\paragraph
\renewcommand{\paragraph}[1]{\oldparagraph{#1}\mbox{}}
\fi
\ifx\subparagraph\undefined\else
\let\oldsubparagraph\subparagraph
\renewcommand{\subparagraph}[1]{\oldsubparagraph{#1}\mbox{}}
\fi

%%% Use protect on footnotes to avoid problems with footnotes in titles
\let\rmarkdownfootnote\footnote%
\def\footnote{\protect\rmarkdownfootnote}


  \title{A communicative framework for early word learning}
    \author{Benjamin C. Morris\textsuperscript{1}~\& Daniel Yurovsky\textsuperscript{1,2}}
    \date{}
  
\shorttitle{Communicative Word Learning}
\affiliation{
\vspace{0.5cm}
\textsuperscript{1} University of Chicago\\\textsuperscript{1} Carnegie Mellon University}
\keywords{keywords\newline\indent Word count: X}
\usepackage{csquotes}
\usepackage{upgreek}
\captionsetup{font=singlespacing,justification=justified}

\usepackage{longtable}
\usepackage{lscape}
\usepackage{multirow}
\usepackage{tabularx}
\usepackage[flushleft]{threeparttable}
\usepackage{threeparttablex}

\newenvironment{lltable}{\begin{landscape}\begin{center}\begin{ThreePartTable}}{\end{ThreePartTable}\end{center}\end{landscape}}

\makeatletter
\newcommand\LastLTentrywidth{1em}
\newlength\longtablewidth
\setlength{\longtablewidth}{1in}
\newcommand{\getlongtablewidth}{\begingroup \ifcsname LT@\roman{LT@tables}\endcsname \global\longtablewidth=0pt \renewcommand{\LT@entry}[2]{\global\advance\longtablewidth by ##2\relax\gdef\LastLTentrywidth{##2}}\@nameuse{LT@\roman{LT@tables}} \fi \endgroup}


\usepackage{lineno}

\linenumbers

\authornote{

Correspondence concerning this article should be addressed to Benjamin C. Morris, Department of Psychology, University of Chicago, 5848 S University Ave, Chicago, IL 60637. E-mail: \href{mailto:yurovsky@uchicago.edu}{\nolinkurl{yurovsky@uchicago.edu}}}

\abstract{
Enter abstract here. Each new line herein must be indented, like this line.


}

\begin{document}
\maketitle

\hypertarget{introduction}{%
\section{Introduction}\label{introduction}}

Word learning as a statistical inference problem.

From Quine on. (Quine, 1960)

three kinds of uncertainty -- over statistical time and in the moment

constraints, pragmatics, etc deal with uncertainty in the moment

uncertainty over consistent meanings -- priors of some kind to deal with this tenenbaum \& xu
(Tenenbaum, 1999,@xu2007)

statistical co-occurrence structure deals with uncerainty reduction over time
(Siskind, 1996,@yu2008,@blythe2010,@blythe2016)

these two scales are linked (Frank, Goodman, \& Tenenbaum, 2009)

linking priors and in the moment scales (Frank \& Goodman, 2012,@frank2014)

All of the arguments in these domains are about the relative difficulty of these different kinds of problems (Trueswell, Medina, Hafri, \& Gleitman, 2013,@smith2014,@yurovsky2014,@yurovsky2015)

but all of this stuff is still about speakers talking to no one! (Tomasello, 2000, @tomasello2001)

Indeed, it looks like it matters whether speech is to children
- structural reasons (Aslin, Woodward, LaMendola, \& Bever, 1996,)
- evidence from weisleder, hoff, etc. (Weisleder \& Fernald, 2013)
- argument from ruthee about structure of contra evidence from Akhtar (Akhtar, Jipson, \& Callanan, 2001,@akhtar2005,foushee2016)

In contrast, pedagogical inference -- shafto, bonawitz, etc. (Bonawitz et al., 2011,@shafto2012)
- evidence for some of this kind of stuff from follow-in labeling. tomasello, baldwin, yu
- but this is probably not what parents are doing most of the time (although c.f. tamis-lemonda) (Tamis-LeMonda, Kuchirko, Luo, Escobar, \& Bornstein, 2017)
- old arguments from newport, etc. (Newport, Gleitman, \& Gleitman, 1977)

An intermediate position: Speakers goal is to communicate
- Grice (1969)

reference games and transmission of language
- Kirby, Tamariz, Cornish, and Smith (2015)
- Gibson et al. (2017)
- Baddeley and Attewell (2009)

Critically, reference games and information theory (in general) assume that speaker and receiver share the same code

But what if only one person knows the code? In this case, in order to communicate successfully, speakers need to take into account the listener's knowledge of the language
- evidence for some speaker design
- brown-schmidt and tanenhaus (Brown-Schmidt, Gunlogson, \& Tanenhaus, 2008)

In this case, ambiguity will be controlled in part by the speaker's communicative goals, and scale with the listener.

We show that without any explicit pedagogical goal, can get speaker design in reference games that leads to better learning

A spectrum of models from pedagogical to adversarial. Figure?

\hypertarget{a-model-of-learning-and-production}{%
\section{A model of learning and production}\label{a-model-of-learning-and-production}}

\hypertarget{brief-explanation-of-the-general-reference-game-framework}{%
\section{Brief explanation of the general reference game framework}\label{brief-explanation-of-the-general-reference-game-framework}}

\hypertarget{experiments-1-and-2}{%
\section{Experiments 1 and 2}\label{experiments-1-and-2}}

speakers adapt to beliefs about points and also speaker knowledge

\hypertarget{method}{%
\section{Method}\label{method}}

\hypertarget{participants}{%
\subsubsection{Participants}\label{participants}}

\hypertarget{material}{%
\subsubsection{Material}\label{material}}

\hypertarget{procedure}{%
\subsubsection{Procedure}\label{procedure}}

\hypertarget{data-analysis}{%
\subsubsection{Data analysis}\label{data-analysis}}

\hypertarget{results}{%
\subsection{Results}\label{results}}

\hypertarget{discussion}{%
\subsection{Discussion}\label{discussion}}

\hypertarget{experiments-3-and-4}{%
\section{Experiments 3 and 4}\label{experiments-3-and-4}}

this leads to better learning, but not as good as ostension (obviously)

\hypertarget{a-model-of-teaching}{%
\section{A model of teaching}\label{a-model-of-teaching}}

\hypertarget{experiment-5}{%
\section{Experiment 5}\label{experiment-5}}

teaching!

\hypertarget{consequences-for-learning}{%
\section{Consequences for Learning}\label{consequences-for-learning}}

In the model and experiments above, we asked whether the pressure to communicate successfully with a linguistically-naive partner would lead to pedagogically supportive input. These results confirmed its' sufficiency: As long as linguistic communication is less costly than deictic gesture, speakers should be motivated to teach in order to reduce future communicative costs. Further, the strength of this motivation is modulated by predictable factors (speaker's linguistic knowledge, listener's linguistic knowledge, relative cost of speech and gesture, learning rate, etc.), and the strength of this modulation is well predicted by a rational model of planning under uncertainty about listner's vocabulary.

In this final section, we take up the consequences of communicatively-motivated teaching for the listener. To do this, we adapt a framework used by Blythe et al. (2010) and colleagues to estimate the learning times for an idealized child learning language under a variety of models of both the child and their parent. We come to these estimates by simulating exposure to successive communicative events, and measuring the probability that successful learning happens after each event. The question of how different models of the parent impact the learner can then be formalized as a question of how much more quickly learning happens in the context of one model than another.

We consider three parent models:

\begin{enumerate}
\def\labelenumi{\arabic{enumi}.}
\item
  \emph{Teacher} - under this model, we take the parents' goal to be maximizing the child's linguistic development. Each communicative event in this model consists of an ostensive labelling event (Note: this model is equivalent to a \emph{Communicator} that ignores communicative cost).
\item
  \emph{Communicator} - under this model, we take the parents' goal to be maximizing communicative success while minimizing communicative cost. This is the model we explored in the previous section.
\item
  \emph{Indifferent} - under this model, the parent produces a linguistic label in each communicative event regardless of the child's vocabulary state. (Note: this model is a special case of the communicator that minimizes communicative cost without seeking to maximize communicative success.)
\end{enumerate}

the child's linguistic development. Each communicative event in this model consists of an ostensive labelling event (Note: this model is equivalent to a \emph{Communicator} who ignores communicative cost).

SOME STUFF ABOUT CROSS SITUATIONAL LEARNING

One important point to note is that we are modeling the learning of a single word rather than the entirety of a multi-word lexicon (as in Blythe et al., 2010). Although learning times for each word could be independent, an important feature of many models of word learning is that they are not (Frank et al., 2009; Yu, 2008; Yurovsky et al., 2014; although c.f. McMurray, 2007). Indeed, positive synergies across words are predicted by the majority of models and the impact of these synergies can be quite large under some assumptions about the frequency with which different words are encountered (Reisenauer, Smith, \& Blythe, 2013). We assume independence primarily for pragmatic reasons here--it makes the simulations significantly more tractable (although it is what our experimental participants appear to assume about learners). Nonetheless, it is an important issue for future consideration. Of course, synergies that support learning under a cross-situational scheme must also support learning from communcators and teachers (Markman \& Wachtel, 1988, @frank2009, @yurovsky2013). Thus, the ordering across conditions should remain unchanged. However, the magnitude of the difference sacross teacher conditions could potentially increase or decrease.

Method

\hypertarget{teaching.}{%
\subsubsection{Teaching.}\label{teaching.}}

Because the teaching model is indifferent to communicative cost, it engages in ostensive an ostensive labeling (pointing + speaking) on each communicative event. Consequently, learning on each trial occurs with a probability that depends entirely on the learner's learning rate (\(P_{k}=p\)). Because we do not allow forgetting, the probability that has failed to successfully learn after \(n\) trials is equal to the probability that they have failed to learn on each of \(n\) successive independent trials (The probabiliy of zero successess on \(n\) trials of a Binomial random variable with parameter \(p\)). The probability of learning after \(n\) trials is thus:

\[ P_k(n) = 1 - \left(1-p\right)^{n}  \]
\#\#\# Communication

\hypertarget{general-discussion}{%
\section{General Discussion}\label{general-discussion}}

\hypertarget{conclusion}{%
\section{Conclusion}\label{conclusion}}

\hypertarget{acknowledgement}{%
\section{Acknowledgement}\label{acknowledgement}}

The authors are grateful to XX and YY for their thoughtful feedback on this manuscript. This research was supported by a James S MacDonnel Foundation Scholars Award to DY.

\newpage

\hypertarget{references}{%
\section{References}\label{references}}

\setlength{\parindent}{-0.5in}
\setlength{\leftskip}{0.5in}

\hypertarget{refs}{}
\leavevmode\hypertarget{ref-akhtar2005}{}%
Akhtar, N. (2005). The robustness of learning through overhearing. \emph{Developmental Science}, \emph{8}(2), 199--209.

\leavevmode\hypertarget{ref-akhtar2001}{}%
Akhtar, N., Jipson, J., \& Callanan, M. A. (2001). Learning words through overhearing. \emph{Child Development}, \emph{72}(2), 416--430.

\leavevmode\hypertarget{ref-aslin1996}{}%
Aslin, R. N., Woodward, J. Z., LaMendola, N. P., \& Bever, T. G. (1996). Models of word segmentation in fluent maternal speech to infants. \emph{Signal to Syntax: Bootstrapping from Speech to Grammar in Early Acquisition}, 117--134.

\leavevmode\hypertarget{ref-baddeley2009}{}%
Baddeley, R., \& Attewell, D. (2009). The relationship between language and the environment: Information theory shows why we have only three lightness terms. \emph{Psychological Science}.

\leavevmode\hypertarget{ref-blythe2016}{}%
Blythe, R. A., Smith, A. D., \& Smith, K. (2016). Word learning under infinite uncertainty. \emph{Cognition}, \emph{151}, 18--27.

\leavevmode\hypertarget{ref-blythe2010}{}%
Blythe, R. A., Smith, K., \& Smith, A. D. M. (2010). Learning times for large lexicons through cross-situational learning. \emph{Cognitive Science}, \emph{34}, 620--642.

\leavevmode\hypertarget{ref-bonawitz2011}{}%
Bonawitz, E., Shafto, P., Gweon, H., Goodman, N. D., Spelke, E., \& Schulz, L. (2011). The double-edged sword of pedagogy: Instruction limits spontaneous exploration and discovery. \emph{Cognition}, \emph{120}(3), 322--330.

\leavevmode\hypertarget{ref-brown-schmidt2008}{}%
Brown-Schmidt, S., Gunlogson, C., \& Tanenhaus, M. K. (2008). Addressees distinguish shared from private information when interpreting questions during interactive conversation. \emph{Cognition}, \emph{107}(3), 1122--1134.

\leavevmode\hypertarget{ref-frank2012}{}%
Frank, M. C., \& Goodman, N. D. (2012). Predicting pragmatic reasoning in language games. \emph{Science}, \emph{336}, 998--998.

\leavevmode\hypertarget{ref-frank2014}{}%
Frank, M. C., \& Goodman, N. D. (2014). Inferring word meanings by assuming that speakers are informative. \emph{Cognitive Psychology}, \emph{75}, 80--96.

\leavevmode\hypertarget{ref-frank2009}{}%
Frank, M. C., Goodman, N., \& Tenenbaum, J. (2009). Using speakers' referential intentions to model early cross-situational word learning. \emph{Psychological Science}, \emph{20}, 578--585.

\leavevmode\hypertarget{ref-gibson2017}{}%
Gibson, E., Futrell, R., Jara-Ettinger, J., Mahowald, K., Bergen, L., Ratnasingam, S., \ldots{} Conway, B. R. (2017). Color naming across languages reflects color use. \emph{Proceedings of the National Academy of Sciences}, \emph{3}, 201619666--6.

\leavevmode\hypertarget{ref-grice1969}{}%
Grice, H. P. (1969). Utterer's meaning and intention. \emph{The Philosophical Review}, \emph{78}, 147--177.

\leavevmode\hypertarget{ref-kirby2015}{}%
Kirby, S., Tamariz, M., Cornish, H., \& Smith, K. (2015). Compression and communication in the cultural evolution of linguistic structure. \emph{Cognition}, \emph{141}, 87--102.

\leavevmode\hypertarget{ref-markman1988}{}%
Markman, E. M., \& Wachtel, G. F. (1988). Children's use of mutual exclusivity to constrain the meanings of words. \emph{Cognitive Psychology}, \emph{20}(2), 121--157.

\leavevmode\hypertarget{ref-mcmurray2007}{}%
McMurray, B. (2007). Defusing the childhood vocabulary explosion. \emph{Science}, \emph{317}(5838), 631--631.

\leavevmode\hypertarget{ref-newport1977}{}%
Newport, E. L., Gleitman, H., \& Gleitman, L. R. (1977). Mother, I'd rather do it myself: Some effects and non-effects of maternal speech style. In C. A. Ferguson (Ed.), \emph{Talking to children language input and interaction} (pp. 109--149). Cambridge University Press.

\leavevmode\hypertarget{ref-quine1960}{}%
Quine, W. V. O. (1960). \emph{Word and object}. \emph{Cambridge, Mass}. Cambridge, Mass.: MIT Press.

\leavevmode\hypertarget{ref-reisenauer2013}{}%
Reisenauer, R., Smith, K., \& Blythe, R. A. (2013). Stochastic dynamics of lexicon learning in an uncertain and nonuniform world. \emph{Physical Review Letters}, \emph{110}(25), 258701.

\leavevmode\hypertarget{ref-shafto2012}{}%
Shafto, P., Goodman, N. D., \& Frank, M. C. (2012). Learning from others the consequences of psychological reasoning for human learning. \emph{Perspectives on Psychological Science}, \emph{7}(4), 341--351.

\leavevmode\hypertarget{ref-siskind1996}{}%
Siskind, J. M. (1996). A computational study of cross-situational techniques for learning word-to-meaning mappings. \emph{Cognition}, \emph{61}, 39--91.

\leavevmode\hypertarget{ref-smith2014}{}%
Smith, L. B., Suanda, S. H., \& Yu, C. (2014). The unrealized promise of infant statistical wordreferent learning. \emph{Trends in Cognitive Sciences}, \emph{18}(5), 251--258.

\leavevmode\hypertarget{ref-tamis-lemonda2017}{}%
Tamis-LeMonda, C. S., Kuchirko, Y., Luo, R., Escobar, K., \& Bornstein, M. H. (2017). Power in methods: Language to infants in structured and naturalistic contexts. \emph{Developmental Science}.

\leavevmode\hypertarget{ref-tenenbaum1999}{}%
Tenenbaum, J. B. (1999). \emph{A bayesian framework for concept learning} (PhD thesis). Massachusetts Institute of Technology.

\leavevmode\hypertarget{ref-tomasello2000}{}%
Tomasello, M. (2000). The social-pragmatic theory of word learning. \emph{Pragmatics}, \emph{10}, 401--413.

\leavevmode\hypertarget{ref-tomasello2001}{}%
Tomasello, M. (2001). Could we please lose the mapping metaphor, please? \emph{Behavioral and Brain Sciences}, \emph{24}(6), 1119--1120.

\leavevmode\hypertarget{ref-trueswell2013}{}%
Trueswell, J. C., Medina, T. N., Hafri, A., \& Gleitman, L. R. (2013). Propose but verify: Fast mapping meets cross-situational word learning. \emph{Cognitive Psychology}, \emph{66}(1), 126--156.

\leavevmode\hypertarget{ref-weisleder2013}{}%
Weisleder, A., \& Fernald, A. (2013). Talking to children matters early language experience strengthens processing and builds vocabulary. \emph{Psychological Science}, \emph{24}(11), 2143--2152.

\leavevmode\hypertarget{ref-xu2007}{}%
Xu, F., \& Tenenbaum, J. B. (2007). Word learning as Bayesian inference. \emph{Psychological Review}, \emph{114}(2), 245--272.

\leavevmode\hypertarget{ref-yu2008}{}%
Yu, C. (2008). A statistical associative account of vocabulary growth in early word learning. \emph{Language Learning and Development}, \emph{4}(1), 32--62.

\leavevmode\hypertarget{ref-yurovsky2015}{}%
Yurovsky, D., \& Frank, M. C. (2015). An integrative account of constraints on cross-situational learning. \emph{Cognition}, \emph{145}, 53--62.

\leavevmode\hypertarget{ref-yurovsky2014}{}%
Yurovsky, D., Fricker, D. C., Yu, C., \& Smith, L. B. (2014). The role of partial knowledge in statistical word learning. \emph{Psychonomic Bulletin \& Review}, \emph{21}, 1--22.

\leavevmode\hypertarget{ref-yurovsky2013}{}%
Yurovsky, D., Yu, C., \& Smith, L. B. (2013). Competitive processes in cross-situational word learning. \emph{Cognitive Science}, \emph{37}, 891--921.


\end{document}
