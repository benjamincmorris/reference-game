\documentclass[english,floatsintext,man]{apa6}

\usepackage{amssymb,amsmath}
\usepackage{ifxetex,ifluatex}
\usepackage{fixltx2e} % provides \textsubscript
\ifnum 0\ifxetex 1\fi\ifluatex 1\fi=0 % if pdftex
  \usepackage[T1]{fontenc}
  \usepackage[utf8]{inputenc}
\else % if luatex or xelatex
  \ifxetex
    \usepackage{mathspec}
    \usepackage{xltxtra,xunicode}
  \else
    \usepackage{fontspec}
  \fi
  \defaultfontfeatures{Mapping=tex-text,Scale=MatchLowercase}
  \newcommand{\euro}{€}
\fi
% use upquote if available, for straight quotes in verbatim environments
\IfFileExists{upquote.sty}{\usepackage{upquote}}{}
% use microtype if available
\IfFileExists{microtype.sty}{\usepackage{microtype}}{}

% Table formatting
\usepackage{longtable, booktabs}
\usepackage{lscape}
% \usepackage[counterclockwise]{rotating}   % Landscape page setup for large tables
\usepackage{multirow}		% Table styling
\usepackage{tabularx}		% Control Column width
\usepackage[flushleft]{threeparttable}	% Allows for three part tables with a specified notes section
\usepackage{threeparttablex}            % Lets threeparttable work with longtable

% Create new environments so endfloat can handle them
% \newenvironment{ltable}
%   {\begin{landscape}\begin{center}\begin{threeparttable}}
%   {\end{threeparttable}\end{center}\end{landscape}}

\newenvironment{lltable}
  {\begin{landscape}\begin{center}\begin{ThreePartTable}}
  {\end{ThreePartTable}\end{center}\end{landscape}}




% The following enables adjusting longtable caption width to table width
% Solution found at http://golatex.de/longtable-mit-caption-so-breit-wie-die-tabelle-t15767.html
\makeatletter
\newcommand\LastLTentrywidth{1em}
\newlength\longtablewidth
\setlength{\longtablewidth}{1in}
\newcommand\getlongtablewidth{%
 \begingroup
  \ifcsname LT@\roman{LT@tables}\endcsname
  \global\longtablewidth=0pt
  \renewcommand\LT@entry[2]{\global\advance\longtablewidth by ##2\relax\gdef\LastLTentrywidth{##2}}%
  \@nameuse{LT@\roman{LT@tables}}%
  \fi
\endgroup}


\ifxetex
  \usepackage[setpagesize=false, % page size defined by xetex
              unicode=false, % unicode breaks when used with xetex
              xetex]{hyperref}
\else
  \usepackage[unicode=true]{hyperref}
\fi
\hypersetup{breaklinks=true,
            pdfauthor={},
            pdftitle={A communicative framework for early word learning},
            colorlinks=true,
            citecolor=blue,
            urlcolor=blue,
            linkcolor=black,
            pdfborder={0 0 0}}
\urlstyle{same}  % don't use monospace font for urls

\setlength{\parindent}{0pt}
%\setlength{\parskip}{0pt plus 0pt minus 0pt}

\setlength{\emergencystretch}{3em}  % prevent overfull lines

\ifxetex
  \usepackage{polyglossia}
  \setmainlanguage{}
\else
  \usepackage[english]{babel}
\fi

% Manuscript styling
\captionsetup{font=singlespacing,justification=justified}
\usepackage{csquotes}
\usepackage{upgreek}

 % Line numbering
  \usepackage{lineno}
  \linenumbers


\usepackage{tikz} % Variable definition to generate author note

% fix for \tightlist problem in pandoc 1.14
\providecommand{\tightlist}{%
  \setlength{\itemsep}{0pt}\setlength{\parskip}{0pt}}

% Essential manuscript parts
  \title{A communicative framework for early word learning}

  \shorttitle{Communicative Word Learning}


  \author{Benjamin C. Morris\textsuperscript{1}~\& Daniel Yurovsky\textsuperscript{1}}

  % \def\affdep{{"", ""}}%
  % \def\affcity{{"", ""}}%

  \affiliation{
    \vspace{0.5cm}
          \textsuperscript{1} University of Chicago  }

  \authornote{
    Correspondence concerning this article should be addressed to Daniel
    Yurovsky, Department of Psychology, University of Chicago, 5848 S
    University Ave, Chicago, IL 60637. E-mail:
    \href{mailto:yurovsky@uchicago.edu}{\nolinkurl{yurovsky@uchicago.edu}}
  }


  \abstract{Enter abstract here. Each new line herein must be indented, like this
line.}
  \keywords{keywords \\

    \indent Word count: X
  }





\usepackage{amsthm}
\newtheorem{theorem}{Theorem}
\newtheorem{lemma}{Lemma}
\theoremstyle{definition}
\newtheorem{definition}{Definition}
\newtheorem{corollary}{Corollary}
\newtheorem{proposition}{Proposition}
\theoremstyle{definition}
\newtheorem{example}{Example}
\theoremstyle{definition}
\newtheorem{exercise}{Exercise}
\theoremstyle{remark}
\newtheorem*{remark}{Remark}
\newtheorem*{solution}{Solution}
\begin{document}

\maketitle

\setcounter{secnumdepth}{0}



\section{Introduction}\label{introduction}

Word learning as a statistical inference problem.

From Quine on. (Quine, 1960)

three kinds of uncertainty -- over statistical time and in the moment

constraints, pragmatics, etc deal with uncertainty in the moment

uncertainty over consistent meanings -- priors of some kind to deal with
this tenenbaum \& xu (J. B. Tenenbaum, 1999,Xu and Tenenbaum (2007))

statistical co-occurrence structure deals with uncerainty reduction over
time (Siskind, 1996,C. Yu (2008),Richard A. Blythe, Smith, and Smith
(2010),Richard A Blythe, Smith, and Smith (2016))

these two scales are linked (Frank, Goodman, \& Tenenbaum, 2009)

linking priors and in the moment scales (Frank \& Goodman, 2012,Frank
and Goodman (2014))

All of the arguments in these domains are about the relative difficulty
of these different kinds of problems (Trueswell, Medina, Hafri, \&
Gleitman, 2013,L. B. Smith, Suanda, and Yu (2014),D. Yurovsky, Fricker,
Yu, and Smith (2014),Daniel Yurovsky and Frank (2015))

but all of this stuff is still about speakers talking to no one!
(Tomasello, 2000, Tomasello (2001))

Indeed, it looks like it matters whether speech is to children -
structural reasons (Aslin, Woodward, LaMendola, \& Bever, 1996,) -
evidence from weisleder, hoff, etc. (Weisleder \& Fernald, 2013) -
argument from ruthee about structure of contra evidence from Akhtar
(Akhtar, Jipson, \& Callanan, 2001,Akhtar (2005),foushee2016)

In contrast, pedagogical inference -- shafto, bonawitz, etc. (Bonawitz
et al., 2011,Shafto, Goodman, and Frank (2012)) - evidence for some of
this kind of stuff from follow-in labeling. tomasello, baldwin, yu - but
this is probably not what parents are doing most of the time (although
c.f. tamis-lemonda) (Tamis-LeMonda, Kuchirko, Luo, Escobar, \&
Bornstein, 2017) - old arguments from newport, etc. (Newport, Gleitman,
\& Gleitman, 1977)

An intermediate position: Speakers goal is to communicate - Grice (1969)

reference games and transmission of language - Kirby, Tamariz, Cornish,
and Smith (2015) - E. Gibson et al. (2017) - Baddeley and Attewell
(2009)

Critically, reference games and information theory (in general) assume
that speaker and receiver share the same code

But what if only one person knows the code? In this case, in order to
communicate successfully, speakers need to take into account the
listener's knowledge of the language - evidence for some speaker design
- brown-schmidt and tanenhaus (Brown-Schmidt, Gunlogson, \& Tanenhaus,
2008)

In this case, ambiguity will be controlled in part by the speaker's
communicative goals, and scale with the listener.

We show that without any explicit pedagogical goal, can get speaker
design in reference games that leads to better learning

A spectrum of models from pedagogical to adversarial. Figure?

\section{A model of learning and
production}\label{a-model-of-learning-and-production}

\section{Brief explanation of the general reference game
framework}\label{brief-explanation-of-the-general-reference-game-framework}

\section{Experiments 1 and 2}\label{experiments-1-and-2}

speakers adapt to beliefs about points and also speaker knowledge

\section{Methods}\label{methods}

\subsubsection{Participants}\label{participants}

\subsubsection{Material}\label{material}

\subsubsection{Procedure}\label{procedure}

\subsubsection{Data analysis}\label{data-analysis}

\subsection{Results}\label{results}

\subsection{Discussion}\label{discussion}

\section{Experiments 3 and 4}\label{experiments-3-and-4}

this leads to better learning, but not as good as ostension (obviously)

\section{A model of teaching}\label{a-model-of-teaching}

\section{Experiment 5}\label{experiment-5}

teaching!

\section{Experiment 6?}\label{experiment-6}

partner game--maybe this waits until next paper?

\section{General Discussion}\label{general-discussion}

\section{Conclusion}\label{conclusion}

\section{Acknowledgement}\label{acknowledgement}

The authors are grateful to XX and YY for their thoughtful feedback on
this manuscript. This research was supported by a James S MacDonnel
Foundation Scholars Award to DY.

\newpage

\section{References}\label{references}

\setlength{\parindent}{-0.5in} \setlength{\leftskip}{0.5in}

\hypertarget{refs}{}
\hypertarget{ref-akhtar2005}{}
Akhtar, N. (2005). The robustness of learning through overhearing.
\emph{Developmental Science}, \emph{8}(2), 199--209.

\hypertarget{ref-akhtar2001}{}
Akhtar, N., Jipson, J., \& Callanan, M. A. (2001). Learning words
through overhearing. \emph{Child Development}, \emph{72}(2), 416--430.

\hypertarget{ref-aslin1996}{}
Aslin, R. N., Woodward, J. Z., LaMendola, N. P., \& Bever, T. G. (1996).
Models of word segmentation in fluent maternal speech to infants.
\emph{Signal to Syntax: Bootstrapping from Speech to Grammar in Early
Acquisition}, 117--134.

\hypertarget{ref-baddeley2009}{}
Baddeley, R., \& Attewell, D. (2009). The relationship between language
and the environment: Information theory shows why we have only three
lightness terms. \emph{Psychological Science}.

\hypertarget{ref-blythe2016}{}
Blythe, R. A., Smith, A. D., \& Smith, K. (2016). Word learning under
infinite uncertainty. \emph{Cognition}, \emph{151}, 18--27.

\hypertarget{ref-blythe2010}{}
Blythe, R. A., Smith, K., \& Smith, A. D. M. (2010). Learning times for
large lexicons through cross-situational learning. \emph{Cognitive
Science}, \emph{34}, 620--642.

\hypertarget{ref-bonawitz2011}{}
Bonawitz, E., Shafto, P., Gweon, H., Goodman, N. D., Spelke, E., \&
Schulz, L. (2011). The double-edged sword of pedagogy: Instruction
limits spontaneous exploration and discovery. \emph{Cognition},
\emph{120}(3), 322--330.

\hypertarget{ref-brown-schmidt2008}{}
Brown-Schmidt, S., Gunlogson, C., \& Tanenhaus, M. K. (2008). Addressees
distinguish shared from private information when interpreting questions
during interactive conversation. \emph{Cognition}, \emph{107}(3),
1122--1134.

\hypertarget{ref-frank2012}{}
Frank, M. C., \& Goodman, N. D. (2012). Predicting pragmatic reasoning
in language games. \emph{Science}, \emph{336}, 998--998.

\hypertarget{ref-frank2014}{}
Frank, M. C., \& Goodman, N. D. (2014). Inferring word meanings by
assuming that speakers are informative. \emph{Cognitive Psychology},
\emph{75}, 80--96.

\hypertarget{ref-frank2009}{}
Frank, M. C., Goodman, N., \& Tenenbaum, J. (2009). Using speakers'
referential intentions to model early cross-situational word learning.
\emph{Psychological Science}, \emph{20}, 578--585.

\hypertarget{ref-gibson2017}{}
Gibson, E., Futrell, R., Jara-Ettinger, J., Mahowald, K., Bergen, L.,
Ratnasingam, S., \ldots{} Conway, B. R. (2017). Color naming across
languages reflects color use. \emph{Proceedings of the National Academy
of Sciences}, \emph{3}, 201619666--6.

\hypertarget{ref-grice1969}{}
Grice, H. P. (1969). Utterer's meaning and intention. \emph{The
Philosophical Review}, \emph{78}, 147--177.

\hypertarget{ref-kirby2015}{}
Kirby, S., Tamariz, M., Cornish, H., \& Smith, K. (2015). Compression
and communication in the cultural evolution of linguistic structure.
\emph{Cognition}, \emph{141}, 87--102.

\hypertarget{ref-newport1977}{}
Newport, E. L., Gleitman, H., \& Gleitman, L. R. (1977). Mother, I'd
rather do it myself: Some effects and non-effects of maternal speech
style. In C. A. Ferguson (Ed.), \emph{Talking to children language input
and interaction} (pp. 109--149). Cambridge University Press.

\hypertarget{ref-quine1960}{}
Quine, W. V. O. (1960). \emph{Word and object}. \emph{Cambridge, Mass}.
Cambridge, Mass.: MIT Press.

\hypertarget{ref-shafto2012}{}
Shafto, P., Goodman, N. D., \& Frank, M. C. (2012). Learning from others
the consequences of psychological reasoning for human learning.
\emph{Perspectives on Psychological Science}, \emph{7}(4), 341--351.

\hypertarget{ref-siskind1996}{}
Siskind, J. M. (1996). A computational study of cross-situational
techniques for learning word-to-meaning mappings. \emph{Cognition},
\emph{61}, 39--91.

\hypertarget{ref-smith2014}{}
Smith, L. B., Suanda, S. H., \& Yu, C. (2014). The unrealized promise of
infant statistical wordreferent learning. \emph{Trends in Cognitive
Sciences}, \emph{18}(5), 251--258.

\hypertarget{ref-tamis-lemonda2017}{}
Tamis-LeMonda, C. S., Kuchirko, Y., Luo, R., Escobar, K., \& Bornstein,
M. H. (2017). Power in methods: Language to infants in structured and
naturalistic contexts. \emph{Developmental Science}.

\hypertarget{ref-tenenbaum1999}{}
Tenenbaum, J. B. (1999). \emph{A bayesian framework for concept
learning} (PhD thesis). Massachusetts Institute of Technology.

\hypertarget{ref-tomasello2000}{}
Tomasello, M. (2000). The social-pragmatic theory of word learning.
\emph{Pragmatics}, \emph{10}, 401--413.

\hypertarget{ref-tomasello2001}{}
Tomasello, M. (2001). Could we please lose the mapping metaphor, please?
\emph{Behavioral and Brain Sciences}, \emph{24}(6), 1119--1120.

\hypertarget{ref-trueswell2013}{}
Trueswell, J. C., Medina, T. N., Hafri, A., \& Gleitman, L. R. (2013).
Propose but verify: Fast mapping meets cross-situational word learning.
\emph{Cognitive Psychology}, \emph{66}(1), 126--156.

\hypertarget{ref-weisleder2013}{}
Weisleder, A., \& Fernald, A. (2013). Talking to children matters early
language experience strengthens processing and builds vocabulary.
\emph{Psychological Science}, \emph{24}(11), 2143--2152.

\hypertarget{ref-xu2007}{}
Xu, F., \& Tenenbaum, J. B. (2007). Word learning as Bayesian inference.
\emph{Psychological Review}, \emph{114}(2), 245--272.

\hypertarget{ref-yu2008}{}
Yu, C. (2008). A statistical associative account of vocabulary growth in
early word learning. \emph{Language Learning and Development},
\emph{4}(1), 32--62.

\hypertarget{ref-yurovsky2015}{}
Yurovsky, D., \& Frank, M. C. (2015). An integrative account of
constraints on cross-situational learning. \emph{Cognition}, \emph{145},
53--62.

\hypertarget{ref-yurovsky2014}{}
Yurovsky, D., Fricker, D. C., Yu, C., \& Smith, L. B. (2014). The role
of partial knowledge in statistical word learning. \emph{Psychonomic
Bulletin \& Review}, \emph{21}, 1--22.






\end{document}
